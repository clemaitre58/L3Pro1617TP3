\documentclass[12pt]{tdtp}
\usepackage{tabularx,colortbl}
\usepackage{multirow}
\usepackage{listings}
\lstset{
	language=VHDL,
basicstyle=\tiny\ttfamily}
\definecolor{light-gray}{gray}{0.96}
\definecolor{pageheading-gray}{gray}{0.2}
\definecolor{dark-gray}{gray}{0.45}
\definecolor{dark-green}{rgb}{0.245,0.121,0.0}

\newcommand{\auteur}{Cedric Lemaitre}
\newcommand{\couriel}{c.lemaitre58@gmail.com}
\newcommand{\promo}{L3 Pro Robotique}
\newcommand{\annee}{2017-2018}
\newcommand{\matiere}{Traitement M3.1}

\newcommand{\tdtp}{TP 3}
\renewcommand{\sujet}{Outils décisionnels}


\begin{document}
\titre
Ce TP propose de réaliser la phase d'entrainement de l'outil décisionnel ainsi que la phase de test\\
\\
\\
\\\

%%%%%%%%%%%
\Exo

À partir des données que vous avez créé lors du Tp3, vous devez réaliser l'entrainement d'un classifieur SVM.
Pour cela, vous devez utiliser la librairie scikit-learn dont vous trouverez la documentation sur internet.
Vous devez utiliser en particulier la fonctionnalité SVC, la fonctionnalité \textit{train} (entraimenent) et \textit{predict}.

%%%%%%%%%%%%
\Exo

Réaliser le même travail pour l'outil décisionnel \textbf{boosting}.
Réaliser la validation cette fois-ci en validation croisée (\textit{cross-validation}

%%%%%%%%%%%
\Exo 

Refaite les 3 premiières questions en utilisant l'outil SIFT pour caractériser vos images
\end{document}
